\documentclass[slides]{pgnotes}

\title{Partitioning}

\begin{document}

\maketitle

\tableofcontents

\section{Block devices}
\label{sec:block-devices}

Each hard disk or hardware RAID set appears as a so-called block device.

\begin{itemize}
\item
  UNIX-like systems show block devices in the \texttt{/dev} directory:

  \begin{itemize}
  \item
    Linux show SATA devices as \texttt{sd} and a letter, such as
    \texttt{/dev/sda}, \texttt{/dev/sdb}.
  \item
    Mac systems show \texttt{disk} followed by a number, like
    \texttt{/dev/disk0}.
  \item
    Varies on other unix-like systems (e.g.~Xenix, AIX, BSD)
  \end{itemize}
\item
  Windows systems will show block devices in the logical disk manager.
  They will \emph{NOT} appear as drive letters!
\end{itemize}

A filesystem is then setup on the block device by formatting it. The
filesystem organises the block device into the familiar structure of
files and folders. Also the filesystem typically handles permissions and
metadata storage.

\section{Partitioning}
\label{sec:partitioning}

Partitioning allows us to divide a single block device into a number of
separate segments that each appear as a separate block device. Each
partition can have a different filesystem on it. In almost all cases
fixed disks in PC-based systems are partitioned.

There are a number of partitioning schemes. A few common concepts:

\begin{itemize}
\item
  A partition table defines one or more partitions on the disk. This is
  located in a known location, usually first block on drive.
\item
  The host's OS needs to boot usually from a disk. Partition scheme
  needs to be compatible with the BIOS or UEFI to achieve this. Also the
  boot loader may need to be set up.
\end{itemize}

Most PC-based systems use either MBR or GPT.

\subsection{Motivation}

\begin{enumerate}
\item \textbf{OS requirements} such as Linux that generally require multiple partitions:
  \begin{itemize}
  \item Boot partition for startup files (\texttt{/boot})
  \item Root partition for main system files (\texttt{/})
  \item Swap partition to support virtual memory (same as Windows page file)
  \end{itemize}
\item \textbf{Multi-booting} different operating systems on a single computer.
  \begin{itemize}
  \item May be different ``instances'' of the same OS.
  \item Less common since virtualisation commonplace but still useful.
  \end{itemize}
\item \textbf{Recovery} partitions enabling a installer or cloning tool to restore a system to a known-good configuration.
  \begin{itemize}
  \item Does depend on the disk itself not failing! (Prefer PXE-based cloning!)
  \end{itemize}

\end{enumerate}

\section{Schemes}

\subsection{Master Boot Record (MBR)}
\label{sec:master-boot-record-mbr}

MBR is the most common PC partitioning scheme:
\begin{itemize}
\item Introduce circa 1983 with DOS.
\item Defines up to four primary partitions on a drive, \autoref{fig:mbr-scheme}.
\end{itemize}

\autoimage{mbr_scheme}{MBR partition scheme}{mbr-scheme}

\textbf{Key points:} 

\begin{itemize}
\item
  MBR layout closely linked with BIOS boot process. One partition marked
  ``active''.
\item
  The MBR scheme stores the partition table at the beginning of the
  first 512 bytes of the drive.

  \begin{itemize}
  \item 446 bytes bootloader
  \item 64 bytes partition table
  \item 2 bytes ``magic number'' (\texttt{0xAA55})
  \end{itemize}
\item
  MBR limited to drives less than approx 2.2 TB.
\end{itemize}

\subsection{Extended Boot Record (EBR)}
\label{sec:extended-boot-record}

EBR allows more than four partitions on a drive:
\begin{itemize}
\item One MBR partition is designated as an extended partition.
\item This \emph{extended partition} can hold one or more \emph{logical partitions}, \autoref{fig:ebr-scheme}.
\end{itemize}

\autoimage{ebr_scheme}{EBR scheme}{ebr-scheme}

\subsection{GUID Partition Table (GPT)}
\label{sec:guid-partition-table}

GUID Partition Table (GPT) is a more modern scheme and is associated with UEFI-based systems.

UEFI is an alternative to BIOS and is used by newer PCs and Apple Macs.
GPT layout is shown in \autoref{fig:gpt-scheme}.

\autoimage{gpt_scheme}{GPT partition table scheme}{gpt-scheme}

\textbf{Points to note:}

\begin{itemize}
\item
  GPT can define up to 128 partitions.
\item
  A so-called \emph{protective MBR} is present in the usual MBR
  location:

  \begin{itemize}
  \item
    Indicates entire disk is a single MBR partition.
  \item
    Shows disk partitioning (and other) software that the MBR scheme is
    not in use.
  \item
    Prevents accidental overwriting by MBR utilities.
  \end{itemize}
\item
  GPT partition table follows protective MBR.
\item
  GPT stores redundant copy of partition table at the end of the drive.
\item
  Boot process more complicated than MBR.
\end{itemize}

\section{Tooling}

\subsection{Windows}

\subsubsection{GUI}

GUI using \textit{Computer Management}, \autoref{fig:logical-disk-manager}.

\autoimage{logical_disk_manager}{Windows Logical Disk Management}{logical-disk-manager}

\subsubsection{PowerShell}

Microsoft are encouraging the use PowerShell in more recent Windows releases to perform system administration tasks:

\begin{itemize}
\item Replicability for aiding remote assistance and scripting
\item Enables partitioning operations to be done remotely over text links
\end{itemize}

\subsection{Linux}

\begin{itemize}
\item Relies on \texttt{parted} command for both MBR and GPT partitions.
\item Other commands such as \texttt{lsblk} for listing also.
\item Partitions will appear as nodes in the \texttt{/dev} directory:
  \begin{itemize}
  \item Precise names will vary even among linux distributions.
  \item Physical disk will appear as \texttt{sd} and letter e.g. \texttt{/dev/sda}
  \item Partitions will appear as \texttt{/dev/sda1} etc. 
  \end{itemize}
\item Same for other UNIX operating systems.
\end{itemize}

\newpage
\subsection{Apple}

\begin{itemize}
\item Macs hava a graphical \textbf{Disk Utility} app, \autoref{fig:disk-utility}.
\item Also can be managed in Bash/zsh using \texttt{diskutil} command. 
\end{itemize}

\newpage
\autoimage{disk_utility}{Disk utility}{disk-utility}

\newpage
\section{Partition operations}

\subsection{Standard operations}

\begin{enumerate}
\item Listing availabile block devices
\item Listing existing partitions and free space
\item Creating a partition
\item Deleting a partition
\end{enumerate}

\subsection{Resizing}

Can shrink or grow a partition:
\begin{itemize}
\item must take account of filesystem using it
\item Generally avoid doing unless you absolutely have to.
\end{itemize}

\newpage
\subsubsection{Shrinking}

Shrinking a partition has two separate operations:
\begin{enumerate}
\item Shrink the filesystem first (see later on!).
\item Then shrink the partition.
\end{enumerate}

\subsubsection{Growing}

Growing a partition similarly consists of two separate operations:
\begin{itemize}
\item Extend the partition.
\item Then extend the filesystem.
\end{itemize}


\end{document}

