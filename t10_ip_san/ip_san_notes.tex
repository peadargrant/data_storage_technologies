\documentclass[slides]{pgnotes}

\title{IP SAN}

\begin{document}

\maketitle

\section{iSCSI}
\label{sec:iscsi}

IP SAN refers to SAN configurations that run over a standard IP network.
\footnote{We don't think about non-IP uses of Ethernet very often. Later
  on, we'll see how a different type of SAN works over ethernet without
  IP. Other specialised protocols exist too: many entertainment venues
  use the Dante protocol to provide low-latency audio connections over
  ethernet.} The most common type of IP SAN is iSCSI:

\begin{itemize}
\item
  iSCSI allows hosts to attach themselves to remote block devices called
  LUNs.\footnote{LUNs are logical unit numbers, a carry-over from SCSI}
  \begin{itemize}
  \item Once attached, they are no different to a physical hard disk, partition or logical volume.
  \end{itemize}
\item
  iSCSI is very easy to get started using.
  \begin{itemize}
  \item It works over standard IP networking and can obviously co-exist with other network usage.
  \item Key software components are free and/or built-in to modern host OSes.
  \end{itemize}
\end{itemize}

\subsection{Components}
\label{sec:components}

A \textbf{host} uses an \textbf{initiator} to attach to a specified
\textbf{target}, . The target may present as one or more \textbf{LUNs}.

\autoimage{iscsi_initiator_target}{iSCSI initiator and target}{iscsi-initiator-and-target}

\subsection{iSCSI Initiator}
\label{sec:iscsi-initiator}

\begin{itemize}
\item
  Once attached, the host is entirely responsible for creating / managing filesystems on the LUN.
  \begin{itemize}
  \item  \emph{This is the key difference to Network Attached Storage (NAS).}
  \end{itemize}
\item
  Initiator can be either:
  \begin{description}
  \item[Software-based:]
    part of the host OS (usually kernel-level driver)
  \item[Hardware-based:]
    controller card or built-in to chipset on host:
    \begin{itemize}
    \item Has \emph{separate} IP configuration to host.
    \item Usually separate physical port.
    \item Can be \emph{converged network adapter (CNA)} where both host networking and storage networking are on same physical port with different IP addresses.
    \end{itemize}
  \end{description}
\end{itemize}

\autoimage{iscsi_offload}{iSCSI offloading}{iscsi-offloading}

\subsection{iSCSI encapsulation}
\label{sec:encapsulation}

iSCSI carries SCSI commands over a standard TCP/IP network, \autoref{fig:iscsi-stack}.

\autoimage{iscsi_stack}{iSCSI stack}{iscsi-stack}

\subsection{iSCSI Protocol data units (PDUs)}
\label{sec:iscsi-pdu}

iSCSI encapsulates SCSI commands within iSCSI protocol data units (PDUs), \autoref{fig:iscsi-pdu}.

The encapsulation can be seen also in terms of how commands are processed, \autoref{fig:iscsi-write}.

\autoimage{iscsi_encapsulation}{iSCSI encapsulation (EMC)}{iscsi-pdu}

\autoimage{iscsi_write}{iSCSI write}{iscsi-write}

\section{iSCSI Targets}
\label{sec:iscsi-targets}

A \textbf{target} presents one or more \textbf{LUNs} to an initiator:
\begin{itemize}
\item Targets can be provided in a number of ways:
  \begin{itemize}
  \item software target on a normal server OS
  \item specialised SAN appliance
  \item Some NAS units have iSCSI target capability also.
  \end{itemize}
\item \textbf{iSCSI initiators connect to targets that present one or more LUNs.}
\item Remember that as a network, an iSCSI storage scheme could encompass
multiple servers drawn from all of the above.
\end{itemize}

\subsection{Dedicated storage appliances}

\begin{itemize}
\item
  Often contain multiple drive holders.
\item
  Can set up one or more RAID sets in multiple levels, possibly with hot spares.
\item
  Can provision logical volumes on RAID sets (thin or thick provisioned)
\item
  Presents as one or more targets, each with one or more LUNs.
\item
  Tightly integrated hardware and custom OS (often Linux/Unix underneath).
\item
  Designed for high availability, hot swap dual PSUs, redundant net interfaces.
\item
  Often has management features like lights-out management card, SNMP support, web and CLI management.
\end{itemize}

\subsection{General-purpose servers}

\begin{itemize}
\item
  iSCSI target daemon can have one or more targets with one or more LUNS
  set up.
\item
  LUNS can point to:

  \begin{description}
  \item[Physical partition]
  or any other block device
  \item[Logical volume]
  (since it's just a block device)
  \item[Disk image file]
  in a number of possible formats
  \end{description}
\end{itemize}


\subsection{NAS units}

Primarily designed for file sharing often offer some iSCSI capabilities:

\begin{itemize}
\item
  Older units tended to use disk image files. More recent units tend to
  use LVM-like setup with iSCSI LUN provisioned as a logical volume.
\item
  Cheap and easy way to experiment with iSCSI before committing to
  financial outlay.
\end{itemize}



\subsection{iSCSI Connection}
\label{sec:connection}

\begin{itemize}
\item
  To connect to a LUN we need to know:  
  The IP address or DNS name of the machine / device providing the target service.
\item
  The name of the target that is presenting the LUN we want.
\item
  Can query list of available targets from server.
\item
  The connection is made at the target level. Once made, all LUNs are
visible to the initiator as block devices.
\end{itemize}


\subsection{Target naming (IQNs)}
\label{sec:target-naming-iqns}

iSCSI targets are named in a systematic way that can seem confusing at
first. Each target has an iSCSI Qualified Name, or IQN. IQNs have 4
parts:

\begin{description}
\item[iqn]
literal string to denote this as an IQN.
\item[naming authority]
usually the reversed domain name of the company (sometimes storage
appliance vendor)
\item[date]
when the naming authority assumed the name
\item[target name string]
defined by the naming authority (the format of these vary across
different target server systems)
\end{description}

An example of an IQN might be:

\begin{itemize}
\item
  \texttt{iqn.2019-12.ie.dkit.staffstorage} is an IQN for a target
  \texttt{staffstorage} by the \texttt{dkit.ie} naming authority who
  assumed control on \texttt{2019-12}.
\end{itemize}

\subsection{Naming authority}

In practice, the naming authority has nothing to do with DNS names other
than by convention:
\begin{itemize}
\item For simple applications, once you know the DNS name / IP of the server and the name of the target, you can normally discover the IQN automatically.
\end{itemize}

\begin{center}
  \textbf{Put simply, just accept that IQNs are strange-looking.}
\end{center}
  
  \textit{As long as you
distinguish one target from another you will not have difficulty.}

\section{Authentication}
\label{sec:username-passwords}

iSCSI provides Challenge-Handshake Authentication Protocol (CHAP) for basic username / password authentication:
\begin{itemize}
\item CHAP secures a target rather than individual LUNs.
\item CHAP is not particularly secure.
\item Still on smaller networks, and to prevent mistakes, it's useful to
  require a username/password to connect to a target.
\end{itemize}

Storage appliances and software target servers normally have the CHAP authentication as an option that can be enabled on a particular target.


\subsection{Mutual authentication}

You can also set up the initiator to require a username/password from the target, known as mutual authentication.

Having a server authenticate to a client can seem odd, but it makes sure you've got the right disk connected and that sensitive data isn't written to an incorrect location.


\section{LUN masking}
\label{sec:lun-masking}

LUN masking is where a target presents only a subset of its LUNs depending on what initiator is connecting.

The LUNs that are masked from a particular initiator do not appear to exist at all.

Support for LUN masking varies amongst target servers and storage appliances.


\section{iSCSI initiator operations}
\label{sec:iscsi-initiator-operations}

We will walk through the process of connecting to a LUN from the point-of-view of the initiator.

We will not worry about how the LUN is provisioned at this point.

Scenario:
\begin{itemize}
\item A target is provisioned on a storage server with IP address \texttt{192.168.1.4}.
\item Naming authority is \texttt{storage.com} starting from December 2019 (\texttt{2019-12}).
\item Target is named \texttt{stuff}.
\item There are two LUNs in this target.
\end{itemize}

\subsection{Linux software initiator}
\label{sec:linux-software-initiator}

There are two parts to iSCSI on Linux:
\begin{enumerate}
\item  the kernel driver itself, and
\item the surrounding software to administer iSCSI connections
  \begin{itemize}
  \item Most linux distributions nowadays use the open-iscsi package
  \end{itemize}
\end{enumerate}
Other UNIX variants like FreeBSD have similar functionality.


\begin{enumerate}
  \newpage
\item
  Host installs the \texttt{open-iscsi} package which provides the
  \texttt{iscsiadm} command. All \texttt{iscsiadm} commands need to be
  run as root (using sudo, omitted from steps below).
\item
  Host runs \textbf{discovery} on the server to see available targets:

\begin{verbatim}
iscsiadm --mode discovery --type sendtargets --portal 192.168.1.4
\end{verbatim}

  Note you can abbreviate mode, type and portal switches as
  \texttt{m,\ t,\ p} as per the man page. The available targets will
  appear (this has just one):

\begin{verbatim}
192.168.1.4:3260,1 iqn.2019-12.com.storage:stuff
\end{verbatim}

  May see a second line for each target if the server is running both
  IPv4 and IPv6.

  \newpage
\item
  Host \textbf{logs in} to the target:

\begin{verbatim}
iscsiadm -m node -T iqn.2019-12.com.storage:stuff -p 192.168.1.4 --login
# note here we use the short m, p switches
\end{verbatim}

  Assuming the login step is successful, the LUNs on the target will
  appear as locally attached SCSI disks. So they will appear in
  \texttt{/dev} directory as \texttt{/dev/sd?} where ? is the drive
  letter.

\item
  The host then can partition and format these disks as if they were
  locally attached physical devices. Usually GPT/MBR single partition.

  \newpage
\item
  Host can disconnect by \textbf{logging out}:

\begin{verbatim}
iscsiadm -m node -T iqn.2019-12.com.storage:stuff -p 192.168.1.4 --logout
\end{verbatim}

  This is like plugging out a disk:
  \begin{itemize}
  \item Therefore the filesystems need to be  unmounted (ejected) before issuing the logout command.
  \end{itemize}

\newpage
\item
  Normally we would enable automatic login and ensure the iscsi service
  is set to start at boot-up.
\end{enumerate}

\subsection{Windows software
initiator}\label{windows-software-initiator}

Windows has a software initiator for iSCSI built-in to the operating
system:
\begin{itemize}
\item This can be accessed graphically or via PowerShell.
\item Once attached to a target, the LUN(s) appear in the Logical Disk Management display and can be partitioned and formatted etc as normal.
\end{itemize}

\subsection{Hardware HBA}\label{hardware-hba}

Hardware HBA allows attachment to iSCSI target without the knowledge of
the host OS.

\begin{itemize}
\item
  Hardware HBA has its own independent connection to the network,
  separate to the host OS. It is a separate interface, with separate IP
  address etc.
\item
  HBA normally appears to the host OS as a SCSI disk controller card.
\item
  Some HBAs have a side-channel to the host OS (often appearing as a
  serial port) for management purposes. Some also include configuration
  tools / software.
\end{itemize}


\end{document}
