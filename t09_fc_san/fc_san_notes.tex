\documentclass[slides]{pgnotes}

\title{Fibre Channel SAN}

\begin{document}

\maketitle

\section{Fibre Channel}
\label{sec:fibre-channel}

\begin{center}
  \textbf{Fibre Channel (FC) SAN} normally involves the use of \textbf{Fibre Channel Protocol (FCP)} over an optical fibre network.
  \end{center}

\subsection{Key concepts}

\begin{itemize}
\item
  Block devices exposed as LUNs on targets that initiators connect to.
\item
  FC normally encapsulates SCSI commands.
\item
  FCP is analogus to IP in ways, and has its own addressing schemes etc.
\item
  Assume that the FC network is solely for FCP stroage traffic.
\item
  FC normally uses hardware HBA.
\end{itemize}

\subsection{FC SAN Components}
\label{sec:fc-san-components}

\begin{description}
\item[Host-Bus Adapter (HBA):] connects the host to the FC SAN.
  \begin{itemize}
  \item Normally appears as a standard disk controller to the operating system
  \item Similar to how hardware RAID controller encapsulates a RAID set
  \item \textit{Note that the fibre channel network is entirely separate from the host's LAN connection(s).}    
  \end{itemize} 
\item[Fabric components:] such as hubs, switches, etc. Specific to Fibre Channel!
\end{description}

\autoimage{fc_hba}{FC HBA}{fc-hba}

\subsection{FC Connections}
\label{sec:fc-connections}

Fibre Channel SAN usually involves optical fibre transmission:

\begin{itemize}
\item
  Both multi-mode (MMF) OM1, OM2, OM3 types and single-mode used.
  \begin{itemize}
  \item Multi-mode used within data centre / suite (under 500m)
  \item Single-mode used for longer distances (over 500m).
  \end{itemize}
\item
  The standard SC, LC, ST connectors are often seen.
\item
  Confusingly, fibre channel can also be carried over copper.
\end{itemize}

\autoimage{fc_connectors}{Fibre Channel Connectors}{fc-connectors}

\autoimage{fc_connectors_hi_density}{Fibre Channel Connectors}{fc-connectors-hi-density}

\section{FC Fabric types}
\label{sec:fc-fabric-types}

\subsection{Point-to-Point}
\label{sec:point-to-point}

The simplest fabric is a point-to-point where a single host node
connects to a single storage node. Both the host and storage nodes ports
are N\_ports (node ports).

\autoimage{fc_point_to_point}{Point-To-Point}{fc-point-to-point}

These are sometimes seen where a single storage appliance has multiple node (N)
ports, each of which is connected to an N port on a host.

\subsection{Switched fabric}
\label{sec:switched-fabric}

Most SAN implementations involve a switched fabric, \autoref{fig:fc-switched-fabric}.

\autoimage{fc_switched_fabric}{FC switched fabric}{fc-switched-fabric}

\subsection{Port types}

Within a switched fabric, ports are designated as different types. Main
types are:

\begin{description}
\item[N\_port (node port)]
connects either to another N\_port as in this example or to an F port on
a switch.
\item[F\_port (fabric port)]
on a switch connects to an N\_port on node.
\item[E\_port (expansion port)]
on a switch connects to an E\_port on another switch.

\begin{itemize}

\item
  Link between two switches is called an \textbf{inter-switch link}
  (ISL).
\end{itemize}
\item[G\_port]
can act as either an \(F\) or \(G\) port.
\end{description}

We say that a fabric has a number of \textbf{tiers} with an \(n\)-tier
fabric requiring traffic to pass through up to \(n\) switches.

\section{FC Layers}
\label{sec:fc-layers}

Fibre Channel is conceptually divided into 5 layers (numbered 4-0):
\begin{description}
\item[FC-4] \textbf{Protocol-mapping} layer, where upper level protocols ULPs such as SCSI, IP, and others are encapsulated into Information Units (IUs).
\item[FC-3] \textbf{Common services} layer, thin layer that could implement functions like encryption or RAID redundancy algorithms; multiport connections; not used.
\item[FC-2] \textbf{Signaling Protocol}, defined by the Fibre Channel Framing and Signaling 4 (FC-FS-5) standard, consists of the low level Fibre Channel network protocols; port to port connections;
\item[FC-1] \textbf{Transmission Protocol}, which implements line coding of signals;
\item[FC-0] \textbf{physical layer}, includes cabling, connectors etc
\end{description}
  
\autoimage{fc_layers}{Fibre Channel Layers}{fc-layers}


\section{Encapsulation}\label{encapsulation}

FC SAN encapsulates SCSI data in FC frames.

\begin{description}
\item[4-bytes start-of-frame]
\item[24-byte frame header]
including source and destination addresses (specific to FC)
\item[2112-byte data field]
consisting of:

\begin{description}
\item[64-byte header (optional)]
\item[2048-byte data payload]
containing SCSI commands (just as with iSCSI in IP SAN)
\end{description}
\item[4-byte CRC error check]
\item[4-byte end-of-frame]
\end{description}


\autoimage{fc_frame}{FC frame}{fc-frame}

\section{FC Addressing}
\label{sec:addressing}

As we can see from the FC frames, we need addresses to identify the
source and destination of FC traffic. Like IP over Ethernet, FC has two
different types of addresses:

\subsection{FC address}
\label{sec:fc-address}

FC addresses are 24-bit dynamically assigned addresses:

\begin{description}
\item[Domain ID (8-bit)]
is a unique number assigned to each switch in fabric. Only 239 addresses
available as some reserved for fabric management services.
\item[Area ID (8-bit)]
denotes a group of ports on one switch (such as a single card)
\item[Port ID (8-bit)]
port within the area ID within the domain ID.
\end{description}


\autoimage{fc_address}{FC address (EMC ISM book)}{fc-address}

\subsection{Fabric address space}

Maximum number of nodes in switched fabric:
\begin{align}
  \mbox{239 domains} \times \mbox{256 areas} \times \mbox{256 ports} & = 15663104 \label{eq:max-number-of-nodes-switched-fabric}
\end{align}

Address assignment:
\begin{itemize}
\item FC address is automatically assigned when an \(N\) port logs on to the fabric.
\item Like DHCP-assigned IP address but deterministic.
\end{itemize}

\subsection{WorldWide Names (WWNs)}
\label{sec:wwn}

WorldWide Names (WWNs) are FC equivalent of MAC addresses:
\begin{itemize}
\item They are considered fixed and normally set/burned in in factory.
\item WWNs are mapped to FC addresses using the Name Server.
\item Opposite to IP networking, we often want to find devices using WWNs which are resolved
to FC addresses.
\end{itemize}

\autoimage{fc_wwn}{FC world wide name (WWN)}{fc-wwn}


\section{Fabric services}
\label{sec:fabric-services}

Each switch provides a number of services at fixed FC addresses (drawn from the reserved range), \autoref{fig:fc-fabric-services}.

\begin{description}
\item[Fabric Login Server (FFFFFE)] node login (connection) to the fabric.
\item[Name Server (FFFFFC)] tracks FC address to WWN associations. Like ARP in IP.
\begin{itemize}
\item
  Synchronises automatically with name servers in other switches in fabric.
\item
  \emph{Formerly called Distributed Name Service but was ambiguous since
  DNS abbreviation means Domain Name Service which is entirely
  different.}
\end{itemize}
\item[Fabric Controller (FFFFFD)]updates nodes and other switches when changes in the fabric occur, likea device or link being added or removed.
\item[Management Server (FFFFFA)]
  allows managment software to monitor and control the operation of the
  fabric.
\end{description}

\autoimage{fc_fabric_services}{FC fabric services (EMC ISM book)}{fc-fabric-services}

\subsection{State Change Notifications (SCN)}

When changes occur the Fabric controller distributes: 
\begin{description}
\item[Registered State Change Notifications (RSCNs)]
  to nodes attached to its own ports.
\item[Switch Registered State Change Notifications (SW-RSCNs)]
  to other switches in the fabric. This causes the name servers to update.
\end{description}


\subsection{FC Login types}
\label{sec:fc-login-types}

\begin{description}
\item[Fabric Login (FLOG)]
when an \(N\) port (on a host / storage device) connects to an \(F\)
port (on a switch):

\begin{itemize}
\item
  During FLOG, WWN of node and port are passed to login server (FFFFFE).
\item
  Login server returns FC address to node.
\item
  Login server registers WWN with FC address so node can be found by
  WWN.
\end{itemize}
\item[Port login (PLOG)]
when an \(N\) port on an initiator makes a connection to the \(N\) port
on a target through the fabric. (Must have completed FLOG first.)
\item[Process Login (PRLI)]
which establishes the Upper Layer Protocol (ULP) (such as SCSI) between
two \(N\) ports.
\end{description}

\section{Zoning}
\label{sec:zoning}

Zoning is where ports in the fabric are logically grouped together, .
Zoning can provide security benefit, avoid accidental mis-configuration
and increase performance. Conceptually like ethernet VLAN.

\autoref{fc_zoning}{FC Zoning}{fc-zoning}

\begin{itemize}
\item
  When zoning is used, a port may be a member of one or more zones.
  Nodes only see other nodes that are same zone as themselves. A zone
  can span multiple switches in the fabric.
\item
  Without zoning, all nodes get all RSCNs. In zoned FC, RSCNs get sent
  only to nodes in zone. Cuts down on the amount of fabric managament
  traffic if done correctly.
\item
  Zone set comprises multiple zones. Can be activated / deactivated as
  single unit within fabric. Only one zone can be active at one time.
\end{itemize}

\subsection{Zone types}
\label{sec:zone-types}

\begin{description}
\item[Port zoning]
where switch ports are grouped into zone(s). (Easy replacement of failed
devices.)
\item[WWN zoning]
where WWNs are grouped into zones. (Allows physical movement.)
\item[Mixed zoning]
combining both. E.g. host WWN and storage port or vice-versa.
\item[Single HBA zoning]
each HBA has own zone created and relevant storage added to it.
\end{description}

\autoimage{fc_zone_types}{FC Zone types (EMC ISM book)}{fc-zone-types}

\section{FC Topologies}
\label{sec:fc-topologies}

\autoimage{fc_topologies}{FC topologies}{fc-topologies}

\subsection{Single-switch}
\label{sec:single-switch}

Single switch topologies are now very common:

\begin{itemize}
\item
  Most FC deployments are in data centre environments where overall
  distance is low.
\item
  Increasing number of ports per switch.
\item
  Cheap and easy to administer. No ISLs, simple fabric.
\end{itemize}

\subsection{Mesh fabric}
\label{sec:mesh-fabric}

A mesh fabric involves multiple switches connected such that each is
connected redundantly to other switches, .

\begin{description}
\item[Full mesh]
where any switch in the fabric is linked to all others directly.
\item[Partial mesh]
where crossing multiple ISLs are necessary to get to some switches.
\end{description}

\autoimage{fc_mesh_topologies}{Mesh topologies}{fc-mesh-topologies}

\subsection{Core-Edge Fabric}
\label{sec:core-edge-fabric}

Core-edge fabrics involve a core tier that connects storage device to
edge switches that connect to host devices, . Many variations on this
theme.

\autoimage{fc_single_core_topology}{Single-core topology (EMC ISM book)}{fc-single-core-topology}


\end{document}

