\documentclass[slides]{pgnotes}

\title{Direct-Attached Storage}
%\label{ch:direct-attached-storage}

\begin{document}

\maketitle

\tableofcontents

\section{Disk service time}
\label{sec:disk-service-time}

The disk service time, $T_S$ is the time taken by a disk to complete an I/O
request, composed of:

\begin{enumerate}
\item
  the seek time, $T$
\item
  average rotational latency, $L$
\item
  data transfer time, $X$
\end{enumerate}

where:

\begin{align}
  T_S = T + L + X
\end{align}

\subsection{Seek time}\label{sec:seek-time}

Seek time is the time required to position the head on the correct
track. Obviously this isn't uniform, so seek time is given separately:

\begin{description}
\item[Full stroke:]
  time taken to move from innermost to outermost track, $T_F$.
\item[Track-to-track:]
  time taken to move between adjacent tracks, $T_T$.
\item[Average:]
  time taken to move head from one random track to another.
\end{description}

We are normally concerned with the average seek time, defined as one-third of the full stroke:

\begin{align}
  \mbox{average seek time} & = \mbox{full-stroke seek time} / 3 \\
  \mbox{full-stroke seek time} & = \mbox{average seek time} * 3
\end{align}

Typical average seek times would range from \SIrange{5}{20}{\milli\second}.

\begin{example}{Seek time}{seek-time}
Calculate the full-stroke seek time for a drive given an average seek time of 6ms.
\tcblower
\begin{align}
  \mbox{full-stroke seek time} & = \SI{6}{\milli\second} \times 3 \\
                               & = \SI{18}{\milli\second}
\end{align}
\end{example}

\subsection{Rotational latency}
\label{sec:rotational-latency}

The average rotational latency, $L$, is the time taken for the drive to revolve half a revolution:

\begin{align}
  L & = 0.5 / \mbox{revolutions per second} \label{eq:rotational-latency}
\end{align}

This measure depends on drive speed, we must convert RPM to revolutions
per second.

\begin{example}{Rotational latency}{rotational-latency}
  Determine the average rotational latency for a 5400-rpm drive:
  \tcblower
\begin{verbatim}
5400 RPM = 5400 / 60 RPs
         = 90 RPs
       L = 0.5 / 90 
         = 5.5 ms
\end{verbatim}
\end{example}

\subsection{Internal transfer time (X)}\label{internal-transfer-time-x}

The data transfer time is how long it takes for one block of data (at a
given size) to be transferred inside the drive.

\begin{align}
  X & = \mbox{block size} / \mbox{internal transfer rate}
\end{align}


\begin{example}{Internal transfer time}{internal-transfer-time}
  Determine the transfer time given an internal transfer rate of 40MB/s and a block size of 32kB.
\tcblower
\begin{verbatim}
X = 32kB / 40 MB/s
  = ( 32 * 1024 ) / ( 40 * 1024 * 1024 )
  = 0.78 ms
\end{verbatim}
\end{example}

\subsection{IOPS}
\label{sec:iops}

Storage performance is commonly quantified in Input/Output operations
Per Second (or IOPS), which is the reciprocal of the disk service time,
TS.

\begin{align}
  \mbox{IOPS} & = 1.0 / T_S \label{eq:ts-to-iops}
\end{align}

\section{Native command queueing}
\label{sec:native-command-queueing}

\begin{minipage}{0.5\linewidth}
  \begin{itemize}
  \item A hard disk receives multiple commands in quick succession. Each command will be delayed by seek time and rotational latency.
  \item SATA Native Command Queueing tries to optimise the overall latency by  re-ordering the commands to reduce these latencies.
  \item All NCQ algorithms will optimise the seek time, but some will also optimise the rotational latency.
  \end{itemize}
\end{minipage}
\begin{minipage}{0.4\linewidth}
\autompimage{ncq_diagram}{Native Command Queueing}{native-command-queueing}
\end{minipage}

\section{Direct-Attached Storage}
\label{sec:direct-attached-storage}

Direct Attached Storage is where a storage device connects 1:1 with host
without a network:

\begin{description}
\item[Host]
is name given to computer system using storage. (Laptop PC, server or
mainframe.)

\begin{itemize}

\item
  Host normally has an internal bus. (PC-based systems: PCI bus).
\end{itemize}
\item[Storage device:]
magnetic disk, solid state drive, tape drive, multi-host storage
appliance.
\item[Host Bus Adapter (HBA)]
connects host to storage device. The HBA \ldots{}

\begin{itemize}
\item
  \ldots{} needs driver support from the host operating system, although
  many use one of a few generic drivers.
\item
  \ldots{} must match interface on the disk side AND the host's
  expansion bus (PCI, ISA, others).
\item
  \ldots{} nowadays often integrated on the host's motherboard.
\end{itemize}
\item[Interface]
defines electrical and communication characteristics between the HBA and
the storage device.

\begin{itemize}

\item
  Common interfaces ATA/IDE/PATA, SATA, SCSI and SAS.
\end{itemize}
\end{description}

\section{Interface types}\label{interface-types}

\begin{description}
\item[ATA/IDE/PATA]
\item[Serial ATA (SATA)]
\item[Small Computer Systems Interface (SCSI)]
\item[Serially Attached SCSI (SAS)]
\end{description}

\subsection{Operating system}\label{operating-system}

\begin{itemize}
\item
  Host operating system will see so-called block devices representing
  each individual disk attached to the HBA.
\item
  Block devices can then be partitioned or otherwise spatially managed
  by the host operating system.
\item
  Operating system normally inserts another layer of abstraction --- the
  filesystem.
\end{itemize}

\section{Multi-device DAS}\label{multi-device-das}

\subsection{Port multipliers}\label{port-multipliers}

Port multipliers can allow a single DAS port to connect to more than one
device:

\begin{minipage}{0.6\linewidth}
\begin{itemize}
\item
  HBA must support port multiplier usage
\item
  Port multiplier is transparent to individual disks
\item
  Bandwidth is shared
\end{itemize}
\end{minipage}
\begin{minipage}{0.4\linewidth}
  \autompimage{sata_multiplier}{SATA Multipler}{sata-multiplier}
\end{minipage}

\subsection{Storage appliances}
\label{sec:storage-appliances}

A storage appliance with multiple disks can connect these to 1 or more
hosts in DAS-type configuration (usually SAS).

\section{Controller utilisation}
\label{sec:controller-utilisation}

The utilisation, $U$, of a disk controller ranges from 0 to 1:

\begin{align}
  0 & <= U <= 1 \label{eq:utilisation-range}
\end{align}

Although we commonly talk about utilisation as a percentage, we must
convert it to a decimal fraction when working out calculations that
involve utilisation. For example, 30\% utilisation would mean that
\texttt{U=0.3}.

\subsection{Average response time}
\label{sec:average-response-time}

The average response time, {\emph{R}}, depends on the disk service time,
taking into account the controller utilisation:

\begin{align}
R & = Ts / ( 1 - U ) \label{eq:response-time}
\end{align}

\begin{example}{Response time under load}{response-time-under-load}
  The disk service time of a disk under no load is 7.8ms.
  What would the response time expected under 70\% load be?
  \tcblower
\begin{align}
R & = \SI{7.8}{\milli\second} / ( 1 - 0.7 ) \\
  & = \SI{26}{\milli\second}
\end{align}
\end{example}

\subsection{IOPS vs utilisation}
\label{sec:iops-vs-utilisation}

A drive will be advertised as capable of doing a certain number of IOPS.
However, if we want to keep response times within a certain limit, we
may limit the number of IOPS to a certain number.

\begin{align}
  \mbox{IOPS limit} = \mbox{U desired} \times \mbox{IOPS advertised} \label{eq:iops-under-load}
\end{align}

\begin{example}{IOPS under load}{iops-under-load}
  A drive is advertised as being capable of 180 IOPS.
  Determine the maximum permissible IOPS if load is limited to 70\%.
\tcblower
\begin{verbatim}
IOPS @ 70%  load = 180 * 0.7
                 = 126 IOPS
\end{verbatim}
\end{example}

\subsection{Multiple disks}\label{multiple-disks}

Sometimes one disk cannot meet the application requirements on its own.
For a given application, the number of disks required, DR, will be
determined by two other quantities:

\begin{itemize}
\item
  Number of disks required to meet the capacity, DC.
\item
  Number of disks required to meet the application IOPS requirement, DI,
  at a given utilisation U.
\end{itemize}

The higher of these two quantities determine DR:

\begin{verbatim}
DR = max ( DC, DI) 
\end{verbatim}

Note that this requirement does not specify at what level the disks are
combined.

\subsection{Meeting IOPS requirement}\label{meeting-iops-requirement}

A common situation is where a given application seeks to guarantee a
minimum number of IOPS with a given controller utilisation time. This
can be done by:

\begin{enumerate}
\item
  Use the controller utilisation to determine the number of available
  IOPS.
\item
  Divide the required IOPs among the number of available IOPS to get the
  minmum number of disks.
\end{enumerate}

\begin{example}{Disk required for IOPS at utilisation}
  A given application has a requirement of 3600 IOPS.
  The disks available have a maximum of 180 IOPS.
  Determine how many disks are required assuming a maximum utilisation of 80\%.
\tcblower
\begin{verbatim}
available IOPS = 180 * 0.8
               = 144
disks required = ceil( 3600 / 144 )
               = 25
\end{verbatim}
\end{example}

\end{document}

