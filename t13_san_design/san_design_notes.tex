\documentclass[slides]{pgnotes}

\title{SAN design}

\begin{document}

\maketitle

\tableofcontents

\section{Purpose}
\label{sec:purpose}

\begin{description}
\item[Online host-specific block storage]
essentially replacing directly-attached block devices on the hosts.
\item[Online clustered block storage]
allowing a number of hosts to share a block device that is formatted
with a clustered/shared file system.
\item[Offline block storage]
replacing removeable drives on hosts. This may be for archival purposes.
\item[Backup]
  where the SAN facilitates access to block devices for backup.
  \begin{itemize}
  \item May be disk-backed LUNs on storage appliance.
  \item May also include tape-based backup systems on SAN-attached tape equipment.
  \end{itemize}
\end{description}

\subsection{Host-specific considerations}

\begin{itemize}
\item
  Will hosts be booting from the SAN?
  \begin{itemize}
  \item There are a number of ways to  provision this.
  \end{itemize}
\end{itemize}

\section{SAN type}
\label{sec:san-type}

\begin{description}
\item[Motivations for IP SAN:] ~ 
  \begin{enumerate}
  \item Low/no-cost entry
  \item Leverage existing IP knowledge
  \item Re-use of infrastructure (incl. WAN)
  \end{enumerate}
\item[Motivations for FC SAN:] ~ 
  \begin{enumerate}
  \item Performance
  \item Reliability
  \item Inherent SAN/LAN demarcation.
  \end{enumerate}
\end{description}

\subsection{Additional considerations}

\begin{itemize}
\item
  Problem-solving technologies available: FCoE, FCoIP, iSCSI-FC
  bridging.
  \begin{itemize}
  \item Unwise to base network around these technologies.
  \end{itemize}
\item
  Depending on organisation structure and culture, there may be
  different teams responsible for storage vs general-purpose networking.
\end{itemize}

\section{High Availability}
\label{sec:high-availability}

SANs hold high-value information critical to continuity of service. A
naively designed SAN may exhibit worse availability characteristics than
simple direct-attached storage on individual hosts.

\subsection{Storage layer availability}
\label{sec:storage-layer-availability}

Storage appliances normally include a number of features to ensure high
availability, \autoref{fig:powervault-rear}.

\autoimage{powervault_rear}{Rear of Dell PowerVault storage appliance}{powervault-rear}

\begin{description}
\item[Disk redundancy]
where volumes / LUNs are provisioned such that the array is redundant
against disk failures. May include RAID, hot spares etc.
\item[Dual power-supply]
where each storage applicance has two Power Supply Units (PSU).
\begin{itemize}
\item
   Each PSU independently capable of supporting the load.
\item
  Data centre environment should have dual power supply to take full advantage of dual PSU.
\end{itemize}
\item[Redundant cooling components] permitting the storage appliance to continue operating despite the failure of 1 (or sometimes more) fans.
  \begin{itemize}
  \item Should be easily replaced.
  \item Often hot swappable.
  \end{itemize}
\item[Dual controller] where the storage array has two controller modules
  (i.e.~high-performance embedded computer).
  \begin{itemize}
  \item Normally works so that either controller can independently provide all services.
  \item May be hot swappable, \autoref{fig:hot-swap-controller}.
  \end{itemize}
\item[Remote management capabilities]
that allow the storage appliance to alert administrators should any
failures occur. Essential that all alerts are enabled and acted upon to
avoid storage appliance remaining in a degraded state (e.g.~failed PSU)
as redundancy will hide this situation.
\end{description}

\autoimage{hot_swap_controller}{Hot-Swap Controller}{hot-swap-controller}

\subsection{Fabric layer availability}
\label{sec:fabric-layer-availability}

The fabric (regardless of type) can represent a significant single point
of failure. Consider the simple SAN layout in \autoref{fig:rabric-no-redundancy}.

\autoimage{fabric_no_redundancy}{Fabric with no redundancy}{fabric-no-redundancy}


This is vulnerable to a failure of the switch SW1. Duplicating the SAN
fabric averts this, \autoref{fig:fabric-with-redundancy}.

\autoimage{fabric_with_redundancy}{Fabric with redundancy}{fabric-with-redundancy}


SAN fabric components (e.g.~switches) should be dual-powered where
possible. This may be via dual-PSU or via external transfer switches.

\subsection{Host layer availability}
\label{sec:host-layer-availability}

Assuming that there is redundancy in the fabric, the host layer requires
approriate redundancy to connect to the fabric.

To transparently utilise the redundant fabric we must ensure that
multipathing is correctly configured.

\subsection{Data centre environment}
\label{sec:data-centre-environment}

A SAN will normally be provisioned within a data centre environment. We
should assume at a minimum:

\begin{description}
\item[Power supply]
is protected by UPS, possibly generator and has diverse A and B power
paths.
\item[Cooling]
is sufficient to meet the load and incorporates redundancy to maintain
operation in the event of a failure.
\item[Fire detection and suppression]
to protect data and ensure service availability.
\item[Physical security]
appropriate to the data and service risk profile.
\end{description}

The provision of a SAN should dictate that the infrastructural services
are sufficient.

\end{document}

