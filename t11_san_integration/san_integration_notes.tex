\documentclass[slides]{pgnotes}

\title{SAN integration}

\begin{document}

\maketitle

\section{SAN integration}
\label{sec:san-integration}

\begin{itemize}
\item  FC SAN primarily is a highly reliable, high-perforamance, data-centre-centric solution that comes at considerable cost.
\item The FCP is non-routable, and is therefored confined spatially.
\item Often we want to broaden access to our existing FC SAN by means of integration technologies.
\end{itemize}

A word of warning:
\begin{itemize}
\item Generally, we shouldn't start out building a SAN that needs to use these primarily.
\item Like WiFi-extenders, poorly thought-out SAN integration can do more harm than good!
\end{itemize}

\section{Fibre Channel over Ethernet (FCoE)}
\label{sec:fibre-channel-over-ethernet-fcoe}

FCoE is where FCP runs over existing ethernet infrastructure:
\begin{itemize}
\item usually on CAT5/6 cabling
\item can coexist with LAN traffic
\end{itemize}
Normally its used to connect ethernet-attached hosts to an existing FC SAN fabric.
\begin{itemize}
\item Layers 1 and 2 are the same we are familiar with from IP networking.
\item Key difference occurs at Layer 3: FCP is encapsulated in ethernet frames.
\end{itemize}

\subsection{FCoE frame format}
\label{sec:fcoe-frame-format}

\begin{itemize}
\item The Ethertype denotes the frame as FCoE (as opposed to IP).
\item \autoref{fig:fcoe-frame} shows the format of a FCoE frame.
\end{itemize}
  
\autoimage{fcoe_frame}{FCoE frame (Wikipedia)}{fcoe-frame}

\subsection{FCoE fabric}
\label{sec:fcoe-fabric}

\begin{itemize}
\item FC natively has a lossless transport.
\item Protocols such as IP running over ethernet normally are tolerant of frames being dropped when congestion is high.
\item FCoE requires extensions to the ethernet protocol to guarantee lossless transmission:
  \begin{itemize}
  \item Called Priority Flow Control.
  \item In practice, need FCoE capable switch.
  \end{itemize}
\end{itemize}  
  

\subsection{FCoE-FC gateway}
\label{sec:fcoe-fc-gateway}

\begin{itemize}
\item A FCoE gateway will connect FCoE devices to an existing FC network, \autoref{fig:fc-gateway}.
  \begin{enumerate}
  \item Acts as a so-called fibre channel forwarder to bridge the FCoE to the FC.
  \item Provides the normal fabric services.
  \end{enumerate}
\item This is actually quite a complex process under the hood:
  \begin{itemize}
  \item Mapping is required from WWNs to MAC addresses.
  \item Gateway device is often a FC switch/director with an FCoE card fitted.
  \end{itemize}
\end{itemize}

\autoimage{fcoe_gateway}{FCoE gateway}{fcoe-gateway}


\subsection{FCoE host hardware}
\label{sec:fcoe-host-hardware}

Hosts traditionally use a FCoE HBA to connect to the FCoE-provided
fabric:
\begin{itemize}
\item Software initiators are theoretically possible, like \texttt{open-fcoe} but are unusual to see in use.
\item Since FCoE and IP LAN traffic run over ethernet, we can converge the two using a FCoE Converged Networking Adapter.
\item Generally recommended that VLANs are used to separate FCoE and IP traffic where they co-exist.
\end{itemize}
  
\autoimage{fcoe_cna}{FCoE converged networking (Wikipedia)}{fcoe-cna}


\subsection{FCoE Application Areas}
\label{sec:fcoe-application-areas}

FCoE's primary purpose is to extend FC SAN by re-using CAT-6 cabling
and/or existing ethernet networking. It would be highly unusual to build
an entire SAN primarily around FCoE.

Although converged networking is beneficial in many applications, FCoE
would not be a primary means of doing so.

FCoE requires gateways, FCoE-aware switches and other expensive
hardware. It is normally not cost effective compared to IP SAN.

FCoE is not routable, and can't cross Layer-2 boundaries. Therefore it's
primarily seen within a data centre or single-building environment.

\textbf{In summary, FCoE is best used to connect hosts to an existing FC
SAN over CAT5/6-carried ethernet where it is impractical to extend the
FC SAN directly to them.}

\section{Fibre Channel over IP (FCIP)}
\label{sec:fibre-channel-over-ip}

Fibre Channel over IP (FCIP) encapsulates fibre channel within IP
packets on an existing IP network. FCIP is normally used to tunnel
directly between two remote FC networks, , and is not used by individual
hosts.

\autoimage{fcip_network}{FC san extended using FCIP (TechTarget)}{fcip-network}

Configuration is generally needed of the two bridges and the network
between them. The encapsulation process does lower performance compared
to native FC or FCoE.

\subsection{FCIP Application areas}
\label{sec:fcip-application-areas}

\textbf{Generally FCIP is appropriate where we need to bridge two
physically remote FC networks into one fabric across a routed IP or WAN
link.} It is our only option when the link involves a WAN or having to
cross Layer-2 boundaries.

If the two FC networks can be connected directly CAT5/6 then FCoE may be
more suitable than FCIP.

Given that the FCIP link is slower than other links in the FC fabric, it
works best where there is comparatively little traffic on the bridged
link. FCIP is therefore good for scenarios like offsite backup,
replication and accessing archival storage.

\section{FC-iSCSI integration}
\label{sec:fc-iscsi-integration}

FC and iSCSI both encapsulate SCSI commands, so conceptally they can be
bridged. This is normally done using a FC to iSCSI gateway, . As iSCSI
runs over TCP/IP it is routable across Layer-2 boundaries.

\autoimage{fc_iscsi_bridge}{iSCSI hosts accessing FC storage via gateway (Cisco)}{fc-iscsi-bridge}

\subsection{Application areas}
\label{sec:application-areas}

\begin{itemize}
\item
  Leveraging existing FC SAN for lower-traffic hosts at minimal cost
  using software initiators or inbuilt HBAs.
\item
  Allowing iSCSI-capable workstations to directly access FC-SAN storage.
\item
  Giving access to FC SAN for servers that need to be located distant
  from the FC SAN.
\end{itemize}

\end{document}
