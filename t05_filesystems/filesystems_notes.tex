\documentclass[slides]{pgnotes}

\title{Filesystems}

\begin{document}

\maketitle

\tableofcontents

\section{Disk-based filesystems}
\label{sec:disk-based-filesystems}

Disk-based file systems interposes between the application layer
(including the user-space OS tools like File Manager) and the block
devices. A file system is created on a block device (= a block device is
formatted with a file system) that is then mounted by the host's
operating system.

\subsection{Common filesystems}
\label{sec:common-filesystems}

As of the present time the most common disk-based filesystems are:

\begin{description}
\item[Microsoft Windows:]
normally NT File System (NTFS) for fixed disks.
\item[Apple Mac OS X:]
APFS (solid-state disks only), HFS+ (formerly all fixed media, remains
on magnetic disks only)
\item[Linux-based]
systems commonly use ext4, but a large range of other systems commonly
appear: ext3, xfs, reiserfs, ZFS*
\item[UNIX-based systems]
tend to have an OS-specific filesystem w/ similar behaviour to linux,
often ZFS
\end{description}

Wikipedia has collated a
\href{https://en.wikipedia.org/wiki/Comparison_of_file_systems}{comparison
of file systems}.

\subsection{Roles of filesystems}\label{sec:roles-of-filesystems}

\begin{description}
\item[Space management]
  using allocation table. Dealing with \emph{fragmentation}.
\item[Naming]
  possibly case ( sensitive \textbar{} insensitive \textbar{} preserving )
\item[Hierarchy]
  using directories.
  \begin{itemize}
    
  \item
    Historically and some specialist systems: \textbf{flat} filesystem
  \end{itemize}
\item[Metadata]
  such as file creation / modification time
\item[Permissions management]
  via metadata to allow ( read \textbar{} write \textbar{} execute ) to
  specific users.
  
  \begin{itemize}
  \item
    Common permissions are owner / group / world (others) in UNIX.
  \item
    The host OS, not the filesystem, actually enforces these restrictions.
  \end{itemize}
\item[Maintaining integrity]
  using redundancy, checksums.
\item[Durable storage]
  by journalling
\end{description}

\section{Journalling}
\label{sec:journalling}

Durability is the key expectation:
\begin{itemize}
\item data once written to filesystem isn't lost (within reasonable assumptions).
\end{itemize}

Journalling is where the intent to change data is recorded to the journal before the actual data itself is modified:. 
\begin{enumerate}
\item
  Data is written to the on-disk journal
\item
  The actual on-disk file data is changed
\item
  The journal is deleted
\end{enumerate}

\autoimage{journalling_filesystem}{Journalling filesystem}{journalling-filesystem}

\subsection{Alternatives to Journalling}
\label{sec:alternatives-to-journalling}

\begin{description}
\item[Copy-on-write:]
where a modification is first written and then the original deleted
\item[Log-structured file system:]
the journal itself is the filesystem
%\item[Soft updates]
\end{description}

\section{Single vs multi-root}
\label{sec:single-vs-multi-root}

Although more an OS-characteristic, there is an important difference between single and multi-root operating systems that becomes relevant for discussion of filesystems:

\begin{description}
\item[Single root] operating systems
  \begin{itemize}
  \item boot with \textbf{one} filesystem mounted as the \textbf{root} \texttt{/} directory.
  \item All others are attached at \textit{mountpoints}  within the root directory.
  \end{itemize}
\item[Multi-root] operating system has multiple roots:
  \begin{itemize}
  \item often indicated by drive-letters e.g. \texttt{C:,\ D:}.
  \item other OSes (e.g. VMS, Novell) use volume labels (e.g. \texttt{SYS})
  \end{itemize}
\end{description}

In general, most UNIX-like OS are single root. Windows is multi-root.


\subsection{Linux file system operations}
\label{sec:linux}

Linux and UNIX tend to use similar commands to those shown here:
\begin{itemize}
\item Always check the precise requirement for your OS.
\end{itemize}
  
\subsection{Creation}
\label{sec:creation}

File system creation, also called Formatting, is done through the \texttt{mkfs} command:

\inputminted{bash}{fs_creation.sh}


\subsection{Mounting}\label{mounting}

Mounting is where a particular a block device formatted with a particular filesystem is attached into the file system hierarchy:
\begin{itemize}
\item A folder, called the ``mount point'' is created on an existing (usually the root) filesystem.
\item Mounting the additional drive then ``covers up'' that folder with root of the mounted drive.
\end{itemize}

\newpage
Consider the example of mounting a block device \texttt{/dev/sdb1} at
the mountpoint \texttt{/mnt/data}:

\inputminted{bash}{fs_mounting.sh}


\subsection{Mounting on boot}\label{mounting-on-boot}

The manual method has some shortcomings:
\begin{itemize}
\item Manual mounts / unmounts persist only as long as the system is running.
\end{itemize}
Instead we'd like to have a configuration file of mountable filesystems.\\
The \href{https://en.wikipedia.org/wiki/Fstab}{\texttt{/etc/fstab}} file provide this:
\begin{itemize}
\item lists filesystems that can be mounted / unmounted without specifying the block device to the \texttt{mount} command
\item additionally can indicate these should be mounted automatically when the system starts up.
\end{itemize}

\newpage
\subsection{/etc/fstab format}

The \texttt{/etc/fstab} file format is whitespace separated of the following fields:

\begin{enumerate}
\def\labelenumi{\arabic{enumi}.}
\item
  \texttt{device-spec} of the device to mount (device name, label, UUID,
  other means to idenitfy)
\item
  \texttt{mount-point} where filesystem can be accessed once mounted
\item
  \texttt{fs-type} identifying the filesystem
\item
  \texttt{options} that may be needed: \texttt{defaults} per-filesystem,
  \texttt{noauto} to not mount on boot
\item
  \texttt{dump} - for use by the \texttt{dump} program for backups
  (backups to be discussed later)
\item
  \texttt{pass} - order for \texttt{fsck} to check errors at boot time:
  0=don't, 1=during, 2=after
\end{enumerate}

\newpage
More advanced generic example, on Wikipedia:

\begin{verbatim}
# device-spec   mount-point     fs-type      options                                          dump pass
LABEL=/         /               ext4         defaults                                            1 1
/dev/sda6       none            swap         defaults                                            0 0
none            /dev/pts        devpts       gid=5,mode=620                                      0 0
none            /proc           proc         defaults                                            0 0
none            /dev/shm        tmpfs        defaults                                            0 0

# Removable media
/dev/cdrom      /mnt/cdrom      udf,iso9660  noauto,owner,ro                                     0 0

# NTFS Windows 7 partition
/dev/sda1       /mnt/Windows    ntfs-3g      quiet,defaults,locale=en_US.utf8,umask=0,noexec     0 0

# Partition shared by Windows and Linux
/dev/sda7       /mnt/shared     vfat         umask=000                                           0 0

# Mounting tmpfs
tmpfs           /mnt/tmpfschk   tmpfs        size=100m                                           0 0

# Mounting cifs
//cifs_server_name/ashare  /store/pingu    cifs         credentials=/root/smbpass.txt            0 0

# Mounting NFS
nfs_server_name:/store    /store          nfs          rw                                        0 0
\end{verbatim}

Simpler sample of \texttt{/etc/fstab} on a Raspberry-Pi:

\begin{verbatim}
proc            /proc           proc    defaults          0       0
PARTUUID=af832fc2-01  /boot           vfat    defaults          0       2
PARTUUID=af832fc2-02  /               ext4    defaults,noatime  0       1
/dev/sdb2             /data           ext4    defaults,noatime  0       0
# a swapfile is not a swap partition, no line here
#   use  dphys-swapfile swap[on|off]  for that
\end{verbatim}

Can often also identify block devices by other means than their
\texttt{/dev} node, for example partition UUID. Avoids ambiguities if
drives are swapped around. Check help for \texttt{mount} command to be
sure.

\subsubsection{Automounters}\label{automounters}

Other means to identify and mount filesystems exist on Linux
(particularly) nowadays: systemd unit files, pmount, automount.

\subsection{Windows}\label{windows}

\subsubsection{Creation}\label{creation-1}

By demonstration through Computer Management.

\subsubsection{Mounting}\label{mounting-1}

Generally a drive letter will be assigned.

Windows can also do a UNIX-style mount to a folder within another
filesystem. Not commonly used though.

\section{Virtual filesystems}
\label{sec:virtual-filesystems}

A filesystem as such is implemented within the operating system normally
in the form of code. This may in fact allow filesystems to be created
that do not directly reside on any block device.

\begin{description}
\item[Network filesystem]
where the host connects to a filesystem located on another computer
using file-sharing protocols like SMB, NFS.
\item[Object storage filesystems]
where the host connects to a storage service that does not itself
implement the normal hierarchical filesystem, but can appear to do so:

\begin{itemize}

\item
  Software to map WebDAV and systems like S3
\end{itemize}
\item[Device filesystem]
usually mounted at \texttt{/dev} used in UNIX-like OSes to provide
access and permission control to peripheral devices (e.g.~HID, serial,
LPT, block devices).
\item[Special filesystems]
seen in UNIX:

\begin{itemize}
\item
  \texttt{sysfs} and \texttt{configfs} kernel
\item
  \texttt{/proc} within info on processes
\end{itemize}
\end{description}

\end{document}

