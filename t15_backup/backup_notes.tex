\documentclass[slides]{pgnotes}

\title{Backup}

\begin{document}

\maketitle

\tableofcontents

\section{Backup}

\subsection{Backup terminology}
\label{sec:backup-terminology}

\begin{description}
\item[Production data]
is actively used in the course of day-to-day operations.
\item[Backup:]
additional copy of production data that is created and retained for the
sole purpose of recovering lost or corrupted data.
\item[Archive:]
store of data that is no longer actively used that is retained on
low-cost secondary storage.
\end{description}

\autoimage{backup_operations}{Backup operations}{backup-operations}

\subsection{Backup purpose}
\label{sec:backup-purpose}

\begin{description}
\item[Disaster recovery]
where production systems are damaged / destroyed.
\item[Operational recovery]
where data loss / corruption occurs due to application failure or user
error (e.g.~important email deleted).
\item[Archival]
for regulatory compliance, historical record, later analysis etc.
\end{description}

\subsection{Backup parameters}
\label{sec:backup-parameters}

\begin{description}
\item[Recovery Point Objective (RPO)]
the most recent point in time to which we should be able to restore.
Dictates backup frequency.
\item[Recovery Time Objective (RTO)]
is the time taken to restore the system to the RPO point after an
incident occurs.
\end{description}

\autoimage{rpo_rto_graph}{RPO-RTO graph}{rpo-rto-graph}

\subsection{Backup methods}
\label{sec:backup-methods}

\begin{description}
\item[Hot / online:]
production system is running during backup:

\begin{itemize}
\item
  Does not interrupt normal operation
\item
  Issues with open files, particularly database backend stores.
\end{itemize}
\item[Cold / offline:]
production system is shut-down for normal operations.
\end{description}

\subsection{Backup scope}
\label{sec:backup-scope}

Scope refers to what is actually backed up:

\begin{itemize}
\item
  Files themselves
\item
  Files and their layout (e.g.~folders, symlinks)
\item
  Metadata (permissions, last updated)
\item
  Is a \emph{Bare Metal Recovery} required?
\end{itemize}

\section{Backup granularity}
\label{sec:backup-granularity}

There are three common granularities of backup, \autoref{fig:backup-granularity}.
\href{https://www.acronis.com/en-us/articles/incremental-differential-backups/}{See
Acronis article} on different granularities for further info.

\autoimage{backup_granularity}{Backup granularity}{backup-granularity}

\subsection{Full backup}
\label{sec:full-backup}

Full backup contains the entire dataset:

\begin{itemize}
\item
  Completeness.
\item
  Frequent full backups may be prohibitive in terms of time and/or
  storage space.
\end{itemize}

\subsection{Incremenetal backup}
\label{sec:incremental-backup}

Incremental backups store only the files changed since the most recent
full or incremental backup was taken. Restoration steps are shown in .

\autoimage{restore_from_incremental}{Restore from incremental}{restore-from-incremental}

\subsection{Cumulative / Differential backup}
\label{sec:cumulative-differential-backup}

Differential backup includes files changed since the last \textbf{full}
backup. Restoration steps are shown in

\autoimage{restore_from_cumalative}{Restore from cumalative}{restore-from-cumalative}

\section{Backup targets}
\label{sec:backup-targets}

Backup data is written to targets. Like any other storage media, targets
may be directly attached to a host or may be LAN or SAN attached.

\subsection{Disk-based backup}
\label{sec:disk-based-backup}

Disk-based backup has low cost, fast backup and recovery. Should be
using monitored RAID to ensure backup availability.

\subsection{Tape-based backup}
\label{sec:tape-based-backup}

Backup traditionally has used tape media.

Tape drives are best operated such that they do not have to start/stop
repeatedly during operation.

Interleaving can be used so that multiple
streams of data can be written to a tape at speed, \autoref{fig:interleave-tape}.

\autoimage{interleave_tape}{Interleave tape}{interleave-tape}

\subsection{Tape libraries}
\label{sec:tape-libraries}

Tape backup can be provisioned on a tape library in medium-large
installations, \autoref{fig:physical-tape-library}.

\autoimage{physical_tape_library}{Physical tape library}{physical-tape-library}

\subsection{Virtual tape library}
\label{sec:virtual-tape-library}

Virtual tape library emulates a tape library but is actually provisioned
usually from SAN LUNs, \autoref{fig:virtual-tape-library}.

\autoimage{virtual_tape_library}{Virtual tape library}{virtual-tape-library}

\subsection{Comparison}

\autoimage{backup_targets_comparison}{Backup targets comparison}{backup-targets-comparison}

\section{Deduplication}

\autoimage{deduplication_at_source}{Deduplication at source}{deduplication-at-source}

\autoimage{deduplication_at_target}{Deduplication at target}{deduplication-at-target}

\end{document}

