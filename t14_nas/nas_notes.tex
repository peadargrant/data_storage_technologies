\documentclass[slides]{pgnotes}

\title{Network-Attached Storage}

\begin{document}

\maketitle

\tableofcontents

\section{Network Attached Storage}\label{network-attached-storage}

NAS allows storage to be shared over a LAN at the filesystem level.

This contrasts to SAN where block level storage is provided.

It is assumed in
a NAS environment that any number of clients can simultaneously connect
to the share.

\autoimage{nas}{NAS}{nas}

\subsection{NAS protocols}
\label{sec:nas-protocols}

\autoref{tab:nas-protocols} summarises the NAS protocols commonly encountered.

\begin{table}[htbp]
  \begin{tabularx}{1.0\linewidth}{l l l X}
    \toprule
    \textbf{Protocol} & ~ & \textbf{Port} & \textbf{Remarks} \\
    \midrule
    Server Message Block & SMB & 135,137-139,445 & Windows-centric but cross-platform. Samba and CIFSd for Linux.\\
    Common Internet File System & CIFS & ~ & Attempted re-name of SMB by Microsoft\\
    Network File System & NFS & ~ & UNIX centric but cross-platform. Multiple versions, currently NFSv4. NFS client on Windows.\\
    Apple Filing Protocol & AFP & 548 & Developed from earlier AppleShare protocol. Not often encountered.\\
    Netware Core Protocol & NCP & 524 & Like SMB has support for print and other sharing as well.\\
    \bottomrule
  \end{tabularx}
  \caption{NAS protocols}
  \label{tab:nas-protocols}
\end{table}

Points to note:

\begin{enumerate}
\item
  Some protocols like SMB/CIFS and NCP were designed for more than file
  sharing. Print sharing, domain and interprocess communication were
  also facilitated. We won't be dealing with these uses here.
\item
  Most NAS protocols run over TCP/IP. Historically other transports were
  used like NetBIOS for SMB and IPX for NCP. Some idiosyncrancies
  remain.
\item
  Although technically routable over WAN links, NAS protocols tended to
  be designed with LAN usage in mind. Often had poor/no security.
\item
  If they need to be used off-site, best to use a VPN link. Tuning the
  configuration can help performance greatly when used over WAN or VPN
  links.
\end{enumerate}

\subsection{Access pattern}
\label{sec:access-pattern}

Network attached storage is normally utilised via a virtual filesystem.
This allows the network filesystem to be mounted (Unix) / mapped as a
drive letter (Windows): Points to note:

\begin{itemize}
\item
  Client sees filesystem of the NAS protocol (SMB, NFS).
\item
  Client is unaware of the underlying filesystem (NTFS, ext4, etc) on
  the server. Also unaware of what block device (local, SAN etc) the
  filesystem resides on.
\item
  Share may be mounted read-only or read-write.
\item
  Permissions are a complex compound of:

  \begin{enumerate}
  \item
    Server filesystem permissions
  \item
    Permissions defined within the share server
  \item
    Client-side permissions
  \end{enumerate}

  Often issues when a Linux/UNIX client connects to an SMB share and
  multiple users on the UNIX client attempt to use the shared drive.
  Configuration can become very tricky!
\end{itemize}

As an alternative there are tools that can connect directly to the
shares like an FTP client, such as \texttt{smbclient} for Linux. Mobile
apps tend to work similarly.

\section{Provisioning}
\label{sec:provisioning}

\subsection{General-purpose server}
\label{sec:general-purpose-server}

General-purpose servers running standard operating systems have server
software installed for most common protocols. General arrangement:

\begin{enumerate}
\item
  One or more shares configured. Each share points to a directory on the
  server's filesystem.
\item
  Server software configured to start on boot.
\item
  Shares on UNIX-like OS are normally configured in the appropriate
  configuration file for the server program. Common examples:

  \begin{enumerate}
  \item
    \texttt{/etc/exports} file defines NFS exports (shares)
  \item
    \texttt{exportfs} command can be used to manage NFS exports
  \item
    \texttt{/etc/smb.conf} or similar Samba configuration file sets up
    SMB shares
  \end{enumerate}
\item
  Windows OS allows configuration of shares from file explorer view or
  PowerShell.
\item
  Storage may be provisioned on a local disk or may be on a LUN from a
  SAN.
\end{enumerate}

\subsection{Storage appliance}
\label{sec:storage-appliance}

A number of vendors supply NAS devices that can, as a minimum provide
SMB and/or NFS services. Often support many other services. Key vendors:
Synology, QNAP. Points to note:

\begin{itemize}
\item
  Often 2 and 4-bay disk setup with RAID and/or LVM.
\item
  Usually built on Linux or FreeBSD internally with a custom web-based
  management application on top.
\item
  Some NAS-type devices also contain iSCSI Target Servers allowing
  SAN-type LUNs to be configured. These are normally backed either by
  LVM volumes or disk images.
\item
  May be possible to connect to a LUN as if it were an internal disk.
\item
  Low-power, quiet and useful in domestic (music, backup, video sharing)
  and light-commercial settings.
\end{itemize}

\subsection{NAS gateway}
\label{sec:nas-gateway}

A NAS gateway / NAS head / protocol converter delivers NAS shares from
SAN volumes:

High-performance embedded dedicated storage appliance that
may have hardware acceleration.

\autoimage{nas_head}{NAS head}{nap-head}

\end{document}

