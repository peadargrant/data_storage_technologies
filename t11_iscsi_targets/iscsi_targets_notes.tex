\documentclass[slides]{pgnotes}

\title{iSCSI targets}

\begin{document}

\maketitle

\section{Target provisioning
operations}\label{target-provisioning-operations}

iSCSI provision requires the \texttt{tgt} server program.
\begin{itemize}
  \item The iSCSI \texttt{tgt} server can expose physical block devices, partitions,
logical volumes (basically anything with an entry in \texttt{/dev}) as a
LUN.
\item It can also expose a disk image file on the filesystem as a LUN.
\end{itemize}

\texttt{tgt} can be installed on Ubuntu/Debian using:

\begin{verbatim}
apt -y install tgt
\end{verbatim}


Imagine our server has LVM set up with a volume group called
\texttt{vg\_main} (possibly amongst others) on it. This VG contains a
number of logical volumes, including \texttt{lv\_data}. We want to
expose \texttt{lv\_data} as a LUN on an iSCSI target. Obviously the
\texttt{lv\_data} volume should not be mounted on the host.

\subsection{Basic target and LUN}\label{basic-target-and-lun}

The \texttt{tgt} server program is controlled from the file
\texttt{iscsi.conf} located in the \texttt{/etc/tgt/conf.d} folder. The
\texttt{iscsi.conf} is almost XML-like and defines one or more targets
with one or more LUNs. Each target section defines one or more LUNs.

This example shows a target \texttt{iqn.2020-03.ie.peadargrant}
containing one LUN backed by the logical volume
\texttt{/dev/vg\_main/lv\_data}:

\begin{verbatim}
<target iqn.2020-03.ie.peadargrant>
    backing-store /dev/vg_main/lv_data
</target>
\end{verbatim}

The iSCSI target server needs to be restarted to pick up any changes:

\begin{verbatim}
sudo systemctl restart tgt
\end{verbatim}

We can use the \texttt{tgtadm} command to print out information about
the running tgt server:

\begin{verbatim}
tgtadm --mode target --op show
\end{verbatim}

Output looks like:

We can then connect to the running tgt server from another machine using
any iSCSI initiator. It doesn't matter whether it's a hardware HBA or
CNA or a software initiator, or what OS is involved.

\subsection{CHAP authentication}\label{chap-authentication}

Our example above could require for example, \texttt{student} and
\texttt{1Password} as follows:

\begin{verbatim}
<target iqn.2020-03.ie.peadargrant>
    backing-store /dev/vg_main/lv_data
    incominguser student 1Password
</target>
\end{verbatim}

If you want mutual authentication, the \texttt{outgoinguser} parameter
works in the same way as \texttt{incominguser} to handle this. Obviously
the initiator will need to be set up correctly too.

\end{document}

