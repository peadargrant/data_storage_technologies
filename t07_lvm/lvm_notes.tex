\documentclass[slides]{pgnotes}

\title{LVM}

\begin{document}

\maketitle

\section{Logical volumes}
\label{sec:logical-volumes}

Logical Volumes can sometimes offer a practical alternative to
partitions:
\begin{itemize}
\item \textbf{Partitioning} can be considered as a \textbf{thick provisioning} of disk space.
\item \textbf{Logical volumes} represent a \textbf{thin provisioning} of disk space.
\end{itemize}
  
Unlike partitoning, logical volumes are OS-dependent and their handling
depends on the host OS in use.

\subsection{Key concepts}

\begin{itemize}
\item
  Volumes are created which appear to OS as a partition, which are then formatted and mounted in the usual way.
\item
  Usually a single disk is used in its entirety:
  \begin{itemize}
  \item Single large partition is created to hold the logical volumes.
  \item Sometimes a small standard boot partition is required.
  \end{itemize}
\item
  Logical Volume manager can resize volumes:
  \begin{itemize}
  \item For this to work non-destructively, file system must support resizing.
  \end{itemize}
\item
  May support snapshot/restore operations.
\end{itemize}

\section{Linux LVM}
\label{sec:linux-lvm}

Linux has an inbuilt Logical Volume Manager (LVM). LVM is built into the
kernel and managed by a number of commands in \texttt{/usr/sbin}.

LVM takes over partitions on physical disks as physical volumes.
Physical volumes are aggregated into volume groups. Logical Volumes,
which act like a partition, are then provisioned on a volume group.

\subsection{Partition layout}
\begin{description}
\item[Physical disks]
appear to the system as block devices.
\item[LVM partitions]
are created on the physical disk, usually 1 for entire disk. 
\end{description}

\subsection{Physical volumes}

Physical volumes map to physical partitions.
Commands dealing with physical volumes start with \texttt{pv}:
\begin{itemize}
\item  Physical volumes are created on a partition using \texttt{pvcreate}.
\item \texttt{pvdisplay} to show information about all physical volumes in the system.
\end{itemize}

\subsection{Volume group}

A volume group aggregates one or more physical volumes to create a virtual block device:
\begin{itemize}
\item This is conceptually the LVM equivalent of a hard disk.
\item The volume group has a name / label, e.g. \texttt{vg01}.
\end{itemize}

Commands dealing with volume groups start with \texttt{vg}:
\begin{itemize}
\item Volume group is created on a Physical volume using \texttt{vgcreate}.
\item Use \texttt{vgdisplay} to show information about volume groups in the system.
\end{itemize}

Once created, the volume group will then appear as a directory under \texttt{/dev}, e.g. \texttt{/dev/vg01}.

\subsection{Logical volume}

A logical volume is provisioned on the volume group, and acts as a virtual partition.

Commands dealing with logical volumes start with \texttt{lv}:
\begin{itemize}
\item Use \texttt{lvdisplay} to show information about logical volumes in the system.
\end{itemize}

Logical volumes will appear in \texttt{/dev} as a subfolder of their volume group:
\begin{itemize}
\item e.g. in the \texttt{/dev/vg01} folder as \texttt{/dev/vg01/vol\_data}.
\end{itemize}


\autoimage{lvm}{LVM}{lvm}


\section{Useful links}\label{useful-links}

\begin{itemize}
\item
  \url{https://www.digitalocean.com/community/tutorials/an-introduction-to-lvm-concepts-terminology-and-operations}
\item
  \url{https://www.server-world.info/en/note?os=Ubuntu_17.04\&p=iscsi\&f=1}
\item
  \url{https://en.wikipedia.org/wiki/Logical_volume_management}
\end{itemize}

\end{document}

