\documentclass[slides]{pgnotes}

\title{Magnetic disks}

\begin{document}
\maketitle
%\label{ch:magnetic-disks}

\section{Components}
\label{sec:components}

\begin{minipage}{0.4\linewidth}
  \autompimage{hard_drive_components_3d}{Hard drive components}{hard-drive-components}
\end{minipage}
\begin{minipage}{0.59\linewidth}
\begin{itemize}
\item
  Cased with controller board, power / data connectors.
\item
  $\ge 1$ aluminimum platters mounted on spindle separated by small gap.
\item
  Motor spins platters at constant speed (5400, 7200 or 10000 RPM)
\item
  Platters (2-sided) coated with magnetically susceptible material.
\item
  Heads mounted on an actuator arm, moves all heads simultaneously.
\item
  Heads separated from platter by small air gap known as the
  ``head flying height'' maintained by a cushion of air.

  \begin{itemize}
  \item
    Before stopping the platter's rotation, the heads are moved towards
    the ``landing zone''.
  \item
    If the heads contact the disk otherwise it is known as a ``head
    crash'' and risks loss of data.
  \end{itemize}
\end{itemize}
\end{minipage}

\section{Geometry}
\label{sec:geometry}

\begin{minipage}{0.4\linewidth}
\autompimage{disk_geometry}{Disk Geometry}{disk-geometry}
\end{minipage}
\begin{minipage}{0.59\linewidth}
\begin{description}

\item[Tracks:]
concentric rings on each platter. Because of multiple platters, each
with two sides, normally refer to cylinder and head to uniquely identify
a track:

\begin{description}
\item[Cylinder:]
identical set of tracks on both surfaces of all drive platters, numbered
from zero at outer edge.
\item[Head:]
head identifies which platter and side that the track identified by the
cylinder is located within.
\end{description}

\item[Sector / block:]
subdivisions of tracks, each numbered from 1.
\end{description}
\end{minipage}

\section{Logical Block Addressing (LBA)}
\label{sec:logical-block-addressing}

In order to be able to locate our data, we need a way to address
sectors:

\begin{description}
\item[Cylinder-Head-Sector (CHS)]
is a 3-part addressing scheme, where the head identifies the platter and
side, the cylinder picks the track and the sector is with reference to
the track. Host needs to be aware of disk geometry:
\item[Logical Block Addressing (LBA)]
is a simple linear addressing scheme, where the drive controller
translates the linear address starting from zero to the 3-part C-H-S
address.
\end{description}

Address in CHS and LBA format are internally binary numbers and often
appear on real systems in hex form. However, they can be understood and
manipulated as integers for familiarisation purposes.

\subsection{Converting CHS to LBA}
\label{sec:converting-chs-to-lba}

The logical block address (LBA) for a given $C, H, S$ tuple corresponding to the cylinder, head and sector is given by the formula:

\begin{align}
  \mbox{LBA} & = ( C \times \mbox{HPC} + H ) \times \mbox{SPT} + ( S - 1 ) \label{eq:chs-to-lba}
\end{align}

where

\begin{itemize}
\item
  HPC is the number of heads per cylinder
\item
  SPT is the number of sectors per track
\end{itemize}

\subsection{Converting LBA to CHS}\label{converting-lba-to-chs}

The LBA to CHS operation requires \textit{three} separate formulas:

\begin{align}
  C & = \mbox{LBA} / ( \mbox{HPC} * \mbox{SPT} ) \label{eq:lba-to-c} \\
  H & = ( \mbox{LBA} / \mbox{SPT} ) \% \mbox{HPC}  \label{eq:lba-to-h} \\
  S & = ( \mbox{LBA} \% \mbox{SPT} ) + 1 \label{eq:lba-to-s}
\end{align}

Noting that:

\begin{itemize}
\item
  The \texttt{/} symbol means integer division where any fractional
  remainder is truncated (NOT rounded)
\item
  The \texttt{\%} symbol means modulo, i.e. the remainder.
\end{itemize}

As a sanity check, remember also that $H$ must be less than the number of
heads per cylinder and that $S$ must be less than or equal to the number
of sectors per track:

\begin{align}
  H & < HPC \label{eq:h-lt-hpc} \\
  S & \le SPT \label{eq:s-le-spt}
\end{align}

\section{Zoned bit recording}
\label{sec:zoned-bit-recording}

Zoned bit recording increases the storage capacity of the disks by
accounting for the varying density of sectors from the spindle to the
edge:

\begin{minipage}{0.69\linewidth}
\begin{itemize}
\item
  \textbf{Cylinders} are grouped into zones based on their distance from the
  spindle.
\item
  The \textbf{zones} are numbered (like cylinders) from zero at the disk edge.
\item
  Suitable number of \textbf{sectors} per cylinder assigned within each zone.
\end{itemize}
\end{minipage}
\begin{minipage}{0.3\linewidth}
\autompimage{zoned_bit_recording}{Zoned bit recording}{zoned-bit-recording}
\end{minipage}

\section{Capacity}
\label{sec:magnetic-disk-capacity}

Disk capacity is normally expressed in bytes, or suitable multiples like \si{\giga\byte}
or \si{\tera\byte}. A confusing situation exists regarding actual vs advertised
capacity.

\begin{itemize}
\item
  Normally when dealing with binary units like bits and bytes, we use
  binary prefixes, taking the multiplier to be 1024.
\item
  Drive manufacturers however like to use a decimal prefixes (multiplier
  of 1000) when advertising, since it leads to higher numbers for the
  same disk capacity.
\end{itemize}

\subsection{Converting from advertised to actual capacity}
\label{sec:converting-from-advertised-to-actual-capacity}

To convert from advertised capacity to actual capacity:

\begin{enumerate}
\item
  Re-write the advertised capacity in bytes by multiplying successively
  by 1000 to remove the decimal prefix, leaving the answer in bytes.
\item
  Successively divide the capacity in bytes by 1024 to re-write using
  binary prefixes
\end{enumerate}

We can write this as a formula:
\begin{align}
  \mbox{actual} & = \frac{ \mbox{advertised} * 1000^x }{ 1024^x } \label{eq:advertised-to-actual-capacity}
\end{align}
where $x$ depends on the units in question:
\begin{verbatim}
kB =>  x = 1
MB =>  x = 2
GB =>  x = 3
TB =>  x = 4
PB =>  x = 5
\end{verbatim}

\subsection{Converting from actual to advertised capacity}
\label{sec:converting-from-actual-to-advertised-capacity}

To convert from actual capacity to advertised capacity:
\begin{enumerate}
\item
  Re-write the actual capacity in bytes by multiplying successively by
  1024 to remove the binary prefix, leaving the answer in bytes.
\item
  Successively divide the capacity in bytes by 1000 to re-write using
  decimal prefixes.
\end{enumerate}

This process can also be written as a formula:
\begin{align}
  \mbox{advertised} & = \frac{\mbox{actual} * 1024^x}{1000^x} 
\end{align}
where $x$ depends on the units in question as above.

\end{document}