\documentclass[slides]{pgnotes}

\title{SAN}

\begin{document}

\maketitle

\section{Storage area networks}

\begin{center}
\textbf{A storage area network provides:}\\
\textbf{hosts} (servers, desktops, workstations) with access to\\
consolidated \textbf{block-level storage}\\
by means of a \textbf{fabric}.
\end{center}


\section{Layers}
\label{sec:ip-san-layers}

SANs are traditionally broken up into conceptual layers, .

\begin{description}
\item[Host layer:]
consisting of the servers / computers that access storage provisioned
through the SAN.
\item[Fabric layer:]
  containing the networking cabling and hardware:
  \begin{itemize}
  \item hubs
  \item switching
  \item routing
  \item protocol converters
  \end{itemize}
\item[Storage layer:]
consisting of storage appliances such as disk arrays, tape drives.
\end{description}

\autoimage{san_architecture}{SAN architecture}{san-architecture}

\autoimage{san_switch}{SAN switch}{san-switch}

\autoimage{storage_array}{Storage array}{storage-array}

\section{SCSI}
\label{sec:scsi}

Small Computer Systems Interface (SCSI) is a protocol used to connect
(mainly) storage devices to a host.

\begin{description}
\item[Parallel SCSI]
using 50-pin and similar connectors
\item[Serially-Attached SCSI (SAS)]
drives are often seen as direct-attached storage in servers.
\end{description}

Because the protocol is well standardised, it often forms the basis of
SAN as it's easy to encapsulate across a network.

\autoimage{external_scsi_connectors}{External SCSI connectors}{external-scsi-connectors}

\section{SAN types}
\label{sec:san-types}

\begin{description}
\item[Fibre Channel]
encapsulates SCSI commands in Fibre-Channel Protocol (FCP) normally
carried on optical fibre. Also Fibre-Channel Over Ethernet (FCoE) and
Fibre-Channel over IP.
\item[iSCSI]
encapsulates SCSI commands in iSCSI protocol on TCP/IP carried over
standard IP network.
\item[Multiheaded-SAS]
\end{description}

\section{Key concepts}
\label{sec:key-concepts}

\begin{itemize}
\item
  Storage appliances (like disk arrays) are set up with logical volumes.
  Due to historical SCSI terminology, these are called \textbf{Logical Unit Numbers (LUNs)}.
\item
  A storage device exposes a \textbf{target} that has one or more logical block devices LUNs associated with it.

  \begin{itemize}
  \item
    Nowadays, most storage appliances can expose more than one target.
    Helps later with access control.
  \end{itemize}
\item
  To access block storage over a SAN, a host's \textbf{initiator} connects to a specific LUN on a specific target:
  \begin{itemize}
  \item
    The initiator is normally a Host-Bus Adapter. Like HBAs for DAS, HBA
    must match host's bus type and the type of SAN being used.
  \item
    Can also have software-initiators for iSCSI and FCOIP.
  \item
    Although iSCSI and FCOIP are IP-based, HBA is separate to host's
    normal networking.
  \item
    The NIC and HBA can be combined on a single card known as a
    Converged Networking Adapter (CNA).
  \end{itemize}
\end{itemize}

\section{Usage points}
\label{sec:usage-points}

\begin{itemize}
\item
  Normally, a LUN can be attached to only one host at a time.
\item
  As a block device, the host OS is entirely in control of the LUN
  (e.g.~formatting).
\item
  With a SAN HBA, the host OS is unaware that the disk is not a
  locally-attached disk:

  \begin{itemize}
  \item
    Boot-from-SAN possible with no local hard disk.
  \item
    SAN can be transparently used on mission-critical legacy systems
    (like Xenix, OS/2).
  \end{itemize}
\item
  SAN can feature in backup configurations.
\end{itemize}

\section{Non-shared storage}
\label{sec:non-shared-storage}

iSCSI conceptually just extends the physical disk drive cable over a
network.

\textbf{Just as we can't normally connect a hard disk to two computers
at the same time, we can't \emph{normally} have two initiators logged
into the same LUN at the same time.}

\begin{itemize}
\item
  No problem detaching a LUN from a host and attaching it to another.
\item
  LUNs that are read-only can be simultaneously attached to multiple hosts.
\item
  So-called clustered filesystems can permit multiple machines access to the same LUN at the same time.
\item Some targets will permit only a single connection to a LUN. Others will
permit multiple connections, and data corruption is bound to occur if
this happens.
\end{itemize}

\section{SAN vs NAS}
\label{sec:san-vs-nas}

A SAN is distinct from NAS in one key respect:

\begin{center}
\textbf{SAN \(\ne\) NAS}
\end{center}

Specifically:
\begin{enumerate}
\item A \textbf{SAN} provisions a remote \textit{block level storage} device for a host.
\item \textbf{NAS} provides a \textit{remote file system}.
\end{enumerate}
SAN and NAS are \emph{complementary} technologies that address \emph{different} use cases.

\end{document}

